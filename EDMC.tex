\documentclass{article}
\usepackage{graphicx}


%\MakePerPage{footnote}
\newtheorem{design}{Design Criteria}
\newtheorem{coro}{Corollary}
\usepackage{hyperref}
\usepackage{amssymb}
\usepackage{amsmath}
%\usepackage[]{algorithm2e}
\usepackage{url}
\usepackage{amsbsy}
\usepackage{amsfonts}
\usepackage{tikz-cd}
\bibliographystyle{unsrt}
\usepackage{caption}
\numberwithin{equation}{section}
\usepackage{amsthm}
\newtheorem{theorem}{Theorem}[section]
\newtheorem{define}{Definition}[section]
\newtheorem{prop}{Proposition}[section]
\newtheorem{corollary}{Corollary}[section]
\newtheorem{principle}{Principle}[section]
\usepackage{graphicx}
\usepackage{amssymb}
\usepackage{float}
\usepackage{url}
\usepackage[utf8]{inputenc}
\usepackage[english]{babel}
%\usepackage[left=3cm,right=3cm,top=2cm,bottom=2cm]{geometry}
\usepackage{fancyhdr}
\usepackage{enumerate}
\usepackage{amssymb}
\usepackage{amsfonts}
\usepackage{amsmath}
\usepackage{mathrsfs}  
\usepackage{faktor}
\usepackage{xfrac}
\DeclareFontFamily{U}{MnSymbolC}{}
\DeclareSymbolFont{MnSyC}{U}{MnSymbolC}{m}{n}
\DeclareMathSymbol{\diamondplus}{\mathbin}{MnSyC}{"7C}
\DeclareMathSymbol{\diamonddot}{\mathbin}{MnSyC}{"7E}
\DeclareFontShape{U}{MnSymbolC}{m}{n}{
	<-6>  MnSymbolC5
	<6-7>  MnSymbolC6
	<7-8>  MnSymbolC7
	<8-9>  MnSymbolC8
	<9-10> MnSymbolC9
	<10-12> MnSymbolC10
	<12->   MnSymbolC12}{}
\def\stat{\mathbf{\Delta}}
\def\ent{\rho_{\vartheta}}
\def\so{\rho_{s^1}}
\def\st{\rho_{s^2}}
\makeatletter
\def\moverlay{\mathpalette\mov@rlay}
\def\mov@rlay#1#2{\leavevmode\vtop{%
		\baselineskip\z@skip \lineskiplimit-\maxdimen
		\ialign{\hfil$\m@th#1##$\hfil\cr#2\crcr}}}
\newcommand{\charfusion}[3][\mathord]{
	#1{\ifx#1\mathop\vphantom{#2}\fi
		\mathpalette\mov@rlay{#2\cr#3}
	}
	\ifx#1\mathop\expandafter\displaylimits\fi}
\makeatother

\newcommand{\cupdot}{\charfusion[\mathbin]{\cup}{\cdot}}
\newcommand{\bigcupdot}{\charfusion[\mathop]{\bigcup}{\cdot}}
\def\ext{\tilde{\mathrm{d}}}

\begin{document}



\date{
	%\textsc{Supervisor:} Supervisor's Name\\[1em]
	\today
}


\title{Entropic Dynamics of Monte Carlo}
%\Title{Global Correlations of Classical and Quantum Systems}

% Author Orchid ID: enter ID or remove command
%\newcommand{\orcidauthorB}{0000-0000-000-000X} % Add \orcidB{} behind the author's name

% Authors, for the paper (add full first names)
\author{Nicholas Carrara $^{1}$}

% Authors, for metadata in PDF

% Affiliations / Addresses (Add [1] after \address if there is only one affiliation.)
%\address{%
% 	$^{1}$ \quad University at Albany; ncarrara@albany.edu\\
% 	$^{2}$ \quad Massachusetts Institute of Technology; kvanslet@mit.edu}

% Contact information of the corresponding author
%\corres{Correspondence: e-mail@e-mail.com; Tel.: (optional; include country code; if there are multiple corresponding authors, add author initials) +xx-xxxx-xxx-xxxx (F.L.)}

% Current address and/or shared authorship
%\firstnote{Current address: Affiliation 3} 
%\secondnote{These authors contributed equally to this work.}
%% The commands \thirdnote{} till \eighthnote{} are available for further notes

%\simplesumm{} % Simple summary

%\conference{} % An extended version of a conference paper
\maketitle
% Abstract (Do not insert blank lines, i.e. \\) 
\abstract{We develop a Monte Carlo method using Entropic Dynamics for solving dynamical systems with some probability distribution $\rho(x)$.}

\section{Introduction}	




\section{Entropic Dynamics}




\section{Entropic Dynamics for Monte Carlo}
We desire a method for updating a probability distribution $\rho(x,t)$ defined over some sample space $\mathbf{X} \subseteq \mathbb{R}^n$ where the distribution necessarily satisfies a continuity equation,
\begin{equation}
\partial_t\rho(x,t) = -\partial_{k}(v^{k}\rho), \qquad k = \{1,\dots,n\},\label{continuity}
\end{equation}
where $\rho v^{k} \stackrel{\mathrm{def}}{=} j^{k}$ is the probability current.  If we can integrate (\ref{continuity}), then we have solved the problem.  For most applications, this will be done by some kind of approximation.  For example, one may throw a sample distribution $\mathcal{X} \subset \mathbf{X}$ according to $j^{k}$ and $\rho$ and numerical integrate using Monte Carlo integration. Whenever the space $\mathbf{X}$ has a large dimension, this method can be computationally expensive.  

Instead of integrating (\ref{continuity}), we will take a different approach and attempt to approximate the expression,
\begin{equation}
\rho(x',t') = \int dx\, P(x',t'|x,t)\rho(x,t),
\end{equation}
where $P(x',t'|x,t)$ is some specified \textit{transition probability}.  This equation is the dual to (\ref{continuity}) and is often referred to as a \textit{Chapman-Kolmogorov} equation.


****

The goal is to determine the set of transition probabilities $P(x'_i|x_i)$ by maximizing the relative entropy,
\begin{equation}
\mathcal{S}[P,Q] = -\sum_{x'_i}P(x'_i|x_i)\log\frac{P(x'_i|x_i)}{Q(x'_i|x_i)},
\end{equation}
subject to the constraints,
\begin{equation}
\sum_{x'_i}P(x'_i|x_i) = 1 \qquad \mathrm{and} \qquad \sum_{x'_i}P(x'_i|x_i)f_n(x'_i,x_i) = \kappa_n(x_i)
\end{equation}
where $\{f_{\ell}\}_{\ell=1}^n$ constitute a set of $n$ functions on the joint space $\mathbf{X}'\times\mathbf{X}$.

****

We construct the transition probability matrix $\mathbf{P}$ as,
\begin{equation}
\mathbf{P} = \begin{bmatrix}
P(x_1'|x_1) & P(x_1'|x_2) & \cdots & P(x_1'|x_n)\\
P(x_2'|x_1) & P(x_2'|x_2) & \dots & P(x_2'|x_n)\\
\vdots & \vdots & \ddots & \vdots\\
P(x_n'|x_1) & P(x_n'|x_2) & \cdots & P(x_n'|x_n)
\end{bmatrix}.
\end{equation}
Then, the update can be written,
\begin{equation}
\mathbf{p}' = \mathbf{P}\mathbf{p},
\end{equation}
or
\begin{equation}
p'_i = {P^j}_ip_j.
\end{equation}


\section{Transition probabilities}



\end{document}